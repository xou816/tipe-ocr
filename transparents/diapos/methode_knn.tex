\clearpage
\slidetitle{Méthode des \textit{k} plus proches voisins}

\begin{slide}

	\begin{itemize}

		\item Alternative très simple aux réseaux de neurones.

		\item On convertit en vecteur l'image du caractère à reconnaître.

		\item L'entier $k$ est fixé. On sélectionne dans la base d'échantillons les $k$ plus proches vecteurs du vecteur image pour la norme euclidienne.

		\item On identifie la classe la plus représentée parmi ces $k$ vecteurs. 

		\item \textbf{Avantage:} pas de phase d'apprentissage, uniquement besoin d'une base d'échantillons.
		\item \textbf{Inconvénient:} il faut avoir une large base d'échantillons accessibles lors de la reconnaissance.

	\end{itemize}

\end{slide}